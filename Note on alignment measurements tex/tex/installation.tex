\section{\label{sec:install} Installation}

\textcolor{red}{\textit{Do we want to say anything else about the installation???}}

\subsection{\label{sec:pulsers} Quadrupole Plate Alignment}
\textcolor{orange}{Manolis}
\medskip

Hand tools were used in the quadrupole plate alignment procedure, see Fig.~\ref{fig:align_hand_tools}. A laser based survey of the plate positions was also done before the quadrupoles were installed into the vacuum chambers, see Section~\ref{sec:laser_survey}.
\begin{figure}[]
	\centering
	\begin{subfigure}{\columnwidth}
		\begin{tikzpicture}
			\draw (0, 0) node[inner sep=0]{\includegraphics[width=0.95\columnwidth]{fig-1-docdb4116.pdf}};
			\draw (0, 0) node[rotate=45,opacity=0.4]{{\fontsize{45}{54}\selectfont\textcolor{red}{\textbf{Preliminary}}}}; % rule of thumb: \fontsize{size}{1.2*size}
		\end{tikzpicture}
		%\includegraphics[width=0.95\columnwidth]{fig-1-docdb4116.pdf}
		\caption{Tool used for determining the top and bottom plate positions.}\label{fig:align_hand_tool_vert}
	\end{subfigure}
	%add desired spacing between images, e. g. ~, \quad, \qquad, \hfill etc (or a blank line to force the subfigure onto a new line).
	\begin{subfigure}{\columnwidth}
		\begin{tikzpicture}
			\draw (0, 0) node[inner sep=0]{\includegraphics[width=0.95\columnwidth]{fig-2-docdb4116.pdf}};
			\draw (0, 0) node[rotate=45,opacity=0.4]{{\fontsize{45}{54}\selectfont\textcolor{red}{\textbf{Preliminary}}}}; % rule of thumb: \fontsize{size}{1.2*size}
		\end{tikzpicture}
		%\includegraphics[width=0.95\columnwidth]{fig-2-docdb4116.pdf}
		\caption{Tool used for determining the outer and inner plate positions.}\label{fig:align_hand_tool_horiz}
	\end{subfigure}
	\caption{\textcolor{orange}{Jason} Hand tools used for determining the quadrupole plate positions \textcolor{red}{\textit{[From DocDB4116; Need to get versions from Wanwei without annotations]}}.}\label{fig:align_hand_tools}
\end{figure}

Whole and half washers (shims) are used to align the quadrupole plates, as the shims adjust the position and angle of the standoffs and plates, see Fig.~\ref{fig:vert_stand_shims}. 
\begin{figure}[]
	\centering
	\begin{subfigure}{\columnwidth}
		\begin{tikzpicture}
			\draw (0, 0) node[inner sep=0]{\includegraphics[width=0.95\columnwidth]{fig-IMG_4849.jpeg}};
			\draw (0, 0) node[rotate=45,opacity=0.4]{{\fontsize{45}{54}\selectfont\textcolor{red}{\textbf{Preliminary}}}}; % rule of thumb: \fontsize{size}{1.2*size}
		\end{tikzpicture}
		%\includegraphics[width=0.95\columnwidth]{fig-IMG_4849.jpeg}
		\caption{A shim being used with a top vertical standoff.}\label{fig:vert_stand_shim_top}
	\end{subfigure}
	%add desired spacing between images, e. g. ~, \quad, \qquad, \hfill etc (or a blank line to force the subfigure onto a new line).
	\begin{subfigure}{\columnwidth}
		\begin{tikzpicture}
			\draw (0, 0) node[inner sep=0]{\includegraphics[width=0.95\columnwidth]{fig-IMG_4847.jpeg}};
			\draw (0, 0) node[rotate=45,opacity=0.4]{{\fontsize{45}{54}\selectfont\textcolor{red}{\textbf{Preliminary}}}}; % rule of thumb: \fontsize{size}{1.2*size}
		\end{tikzpicture}
		%\includegraphics[width=0.95\columnwidth]{fig-IMG_4847.jpeg}
		\caption{A shim being used with a bottom vertical standoff.}\label{fig:vert_stand_shim_bot}
	\end{subfigure}
	\caption{Whole and half washers (shims) are used to align a Mylar Q1 outer plate.}\label{fig:vert_stand_shims}
\end{figure}

Extra care was taken with the alignment of the Q1 long and short quadrupoles due to the wrinkles in the Q1 outer plate Mylar windows, see~Fig.~\ref{fig:plate_align_trolley}.
\begin{figure}[]
	\centering
	\begin{tikzpicture}
		\draw (0, 0) node[inner sep=0]{\includegraphics[width=0.95\columnwidth]{fig-IMG_4842.jpeg}};
		\draw (0, 0) node[rotate=45,opacity=0.4]{{\fontsize{45}{54}\selectfont\textcolor{red}{\textbf{Preliminary}}}}; % rule of thumb: \fontsize{size}{1.2*size}
	\end{tikzpicture}
	%\includegraphics[width=0.95\columnwidth]{fig-IMG_4842.jpeg}
	\caption{The trolly was used in the alignment of the Q1 long and short quadrupoles.}\label{fig:plate_align_trolley}
\end{figure}

Glue was used to get rid of the gaps between the outer/inner plates and horizontal standoffs, see Fig.~\ref{fig:glue_horiz_stand}. This was done to reduce the number of electrical breakdowns at the Macor-aluminum interfaces. 
\begin{figure}[]
	\centering
	\begin{tikzpicture}
		\draw (0, 0) node[inner sep=0]{\includegraphics[width=0.95\columnwidth]{fig-IMG_4676.jpeg}};
		\draw (0, 0) node[rotate=45,opacity=0.4]{{\fontsize{45}{54}\selectfont\textcolor{red}{\textbf{Preliminary}}}}; % rule of thumb: \fontsize{size}{1.2*size}
	\end{tikzpicture}
	%\includegraphics[width=0.95\columnwidth]{fig-IMG_4676.jpeg}
	\caption{Outer and inner plate horizontal standoffs were glued to the plates.}\label{fig:glue_horiz_stand}
\end{figure}

\medskip
\textcolor{red}{\textit{Hogan needs to comment on his standoff gluing practices.}}

\subsection{\label{sec:laser_survey} In Situ Quadrupole Plate Survey}
\textcolor{orange}{Manolis}
\medskip
