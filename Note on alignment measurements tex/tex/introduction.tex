\section{\label{sec:intro} Introduction}
\medskip

% Give introduction of EQS.

The g-2 storage ring acts as a weak-focusing betatron, with the vertical focusing provided by the Electrostatic Quadrupole System (EQS)~\cite{Semertzidis:2003zs}. A pure quadrupole electric field provides a linear restoring force in the vertical direction, and the combination of the (defocusing) electric field and the central (dipole) magnetic field provides a net linear restoring force in the radial direction. 

Ideally, the EQS plates should fill as much of the azimuth as possible, but space is required for the inflector and kicker magnets, fiber monitors, and field trolley garage. 
Both BNL and Fermilab Muon g-2 experiments chose a symmetric configuration which places the quadrupoles in four distinct regions, Q1-Q4, as shown in Fig.~\ref{fig:ring_schematic}.
Gaps at \ang{0} and \ang{90} for the inflector and kicker magnets, along with empty gaps at \ang{180} and \ang{270} provide a four fold symmetry. 


\begin{figure}[hb]
	\centering
	\includegraphics[width=0.80\textwidth]{ring_schematic.png}
	\caption{A schematic overview of the \gmtwo storage ring. The locations of the 4 EQS sections are marked in red.}\label{fig:ring_schematic}
\end{figure}



Overall the electrodes occupy 43\% of the total circumference. The four-fold symmetry keeps the variation in the beta function small, which minimizes beam breathing and improves the muon orbit stability. Each quad segment consists of a short quad with azimuthal coverage of \ang{13} and a long quad of \ang{26}, for two reasons: 1) to make every quadrupole half-segment independent of others, facilitating their development, testing, etc., and 2) to reduce the extent of low-energy electron trapping. Therefore, there are two high-voltage vacuum-to-air interfaces for each quadrupole segment~\cite{Grange:2015fou}. 


A schematic representation of a cross-section of the electrostatic quadrupoles is shown in Fig.~\ref{fig:electrode_endview} with the various dimensions indicated. 
The four aluminum plates are symmetrically placed around the muon storage region, with \SI{10}{cm} between opposite plates. 



\subsection*{Plate alignment}
% Talk about importance of alignment and exact knowledge of plates' location.

Any misalignment of ESQ plates translates to a modification of their E-field and thus a deviation from ideal storage conditions, inducing perturbations to the orbit of stored muons and systematic shifts to the experimental measurement. 
Therefore good knowledge of the location of the ESQ plates is necessary for reliable operation and analysis.



\begin{wrapfigure}{rt}{0.6\textwidth}
	\centering
	\includegraphics[width=0.56\textwidth]{electrode_endview.png}
	\caption{A cross-sectional schematic view of the four electrodes, referred to as inner, outer, top, and bottom plates.}\label{fig:electrode_endview}
\end{wrapfigure}



The placement accuracy specifications for the EQS plates are determined by the estimated effects of a potential misplacement on the electric field and the muon orbit. For the vertically-installed side electrodes (inner and outer plates) the average deviation from the ideal position should be within \SI{0.75}{mm}. For the top and bottom plates the average deviation should be within \SI{0.5}{mm}. 
% Why is the spec lower for the top/bottom plates? I don't think it's because they are more important for teh field/orbit. It may be because they are much easier to install accurately than the inner/outer.
These are the same specifications as were used E821, and are retained for the Fermilab Muon g-2 Experiment.
We further state that the standard deviation (RMS) over a short quad length should be less than $\pm$\SI{0.75}{mm} for the inner/outer plates, and less than $\pm$\SI{0.5}{mm} for the top/bottom plates.
The deviation from the ideal location should not exceed $\pm$\SI{2}{mm} at any point~\cite{Grange:doc3632}, a specification that is important also for safe clearance from the NMR trolley.
%% reference for these specs? This is apparently based on "the E821 and Bill?s calculation"


%% Have paragraph on potential size of effect from misalignment?
To give a scale of the potential effect, the size of the E-field correction can be expressed as~\cite{Fienberg:thesis}
\begin{equation} \label{eq:Dwa_Wa}
\frac{\Delta\omega_{\alpha}}{\omega_{\alpha}} = \frac{2 \beta^2 n (1-n)}{R_0^2} (\langle x_{eq}^2 \rangle + \langle x_{eq} \rangle \frac{h}{1-n}) ,
\end{equation}
where $\langle x_{eq} \rangle$ is the average offset of the equilibrium orbit from the center of the quadrupole potential, and $h$ is the displacement of the center of the quadrupole potential.
Given the large offset of the equilibrium orbit in Run 1 of approximately \SI{5}{mm}, a \SI{1}{mm} displacement of the quadrupole plates can shift the correction by approximately \SI{20}{ppb}.
An independent tracking study~\cite{Zhanibek:doc13939} finds roughly 10-\SI{25}{ppb} shift in the E-field correction per mm of quad displacement, confirming the size of the effect.
Given the linear coupling with the offset in the equilibrium radius distribution, a 1 mm uncertainty in the quadrupole potential center would similarly contribute $\sim$\SI{20}{ppb} of uncertainty to the correction.
Therefore sub-mm precision is desired in our knowledge of the plate location.








