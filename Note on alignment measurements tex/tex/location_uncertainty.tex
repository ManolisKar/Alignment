\section{\label{sec:location_uncertainty} Location and uncertainty of quadrupole plates }
\medskip


Combining our knowledge from the three different surveys, which are overall in very good agreement, we arrive at our best estimation for the location of EQS plates and estimate the uncertainty.


For the "known" quads, apart from Q1S and Q4S which were reinstalled, the laser scan data are the best estimation and define their location.
The uncertainty can be extracted from the high-precision measurement with the extension through the flange. From the \textit{Diff. to laser scan} column in Table~\ref{tab:Keyence_data}, that difference averages less than \SI{0.5}{mm} per plate. We will use that conservatively as the baseline for each plate's location uncertainty. 
We may increase that baseline uncertainty for the outer plates, which were not sampled. We use \SI{0.7}{mm}, which is the largest entry in that column, even if in comparison to Q1S laser scan data before the vacuum incident. 



\begin{table}[]
\begin{center}
\caption{Radial position and uncertainty contributions for each quad plate. All values in mm. }

\begin{tabularx}{\textwidth}{Y >{\centering\arraybackslash}p{0.065\textwidth}| >{\centering\arraybackslash}p{0.10\textwidth} >{\centering\arraybackslash}p{0.155\textwidth} >{\centering\arraybackslash}p{0.14\textwidth} >{\centering\arraybackslash}p{0.085\textwidth}>{\centering\arraybackslash}p{0.17\textwidth}}
\toprule

    &    &   \multirow{ 2}{*}{\textbf{Avg r}}    &    \small{Difference{\newline}Scan-Extens.}    &    \small{Difference {\newline}Scan-VCNL}    &    \small{Unknown{\newline} plates}    &    \textbf{Total{\newline} uncertainty}  \\ \midrule
 {\multirow{3}{*}{\textbf{Q1S}}} &  \small{Inner}    &    \textbf{7061.58}    &    0.7    &    0.5    &    0.5    &    \textbf{0.99}  \\ 
 {}                                              & \small{Outer}    &    \textbf{7161.44}    &    0.7    &    0.5    &    0.5    &   \textbf{0.99}    \\  \midrule
 {\multirow{2}{*}{\textbf{Q1L}}} &  \small{Inner}    &    \textbf{7061.47}    &    0.5    &    0.18    &            & \textbf{0.53}  \\ 
 {}                                              & \small{Outer}    &    \textbf{7162.04}    &    0.7    &    0.44    &            &  \textbf{0.83}    \\  \midrule
 {\multirow{2}{*}{\textbf{Q2S}}} &  \small{Inner}    &    \textbf{7061.30}    &    0.5    &    0.01    &             &  \textbf{0.50}  \\ 
 {}                                              & \small{Outer}    &    \textbf{7160.71}    &    0.7    &    0.05    &            &  \textbf{0.70}    \\  \midrule
 {\multirow{2}{*}{\textbf{Q2L}}} &  \small{Inner}    &    \textbf{7062.06}    &    0.5    &    0.29    &            &   \textbf{0.58}  \\ 
 {}                                              & \small{Outer}    &    \textbf{7161.87}    &    0.7    &     0.11   &             &    \textbf{0.71}    \\  \midrule
 {\multirow{2}{*}{\textbf{Q3S}}} &  \small{Inner}    &    \textbf{7061.60}    &    0.5    &    0.08    &           &    \textbf{0.51}  \\ 
 {}                                              & \small{Outer}    &    \textbf{7162.00}    &    0.7    &    0.41    &           &   \textbf{0.81}    \\  \midrule
 {\multirow{2}{*}{\textbf{Q3L}}} &  \small{Inner}    &    \textbf{7062.31}    &    0.5    &    0.08    &            &    \textbf{0.51}  \\ 
 {}                                              & \small{Outer}    &    \textbf{7161.92}    &    0.7    &    0.13    &           &  \textbf{0.71}    \\  \midrule
 {\multirow{2}{*}{\textbf{Q4S}}} &  \small{Inner}    &    \textbf{7061.61}    &    0.7    &    0.5    &    0.5    &    \textbf{0.99}  \\ 
 {}                                              & \small{Outer}    &    \textbf{7161.70}    &    0.7    &    0.5    &    0.5    &    \textbf{0.99}    \\  \midrule
 {\multirow{2}{*}{\textbf{Q4L}}} &  \small{Inner}    &    \textbf{7062.20}     &    0.5    &    0.44    &            &   \textbf{0.67}  \\ 
 {}                                              & \small{Outer}    &    \textbf{7161.97}    &    0.7    &    0.39    &            &   \textbf{0.80}    \\  
\bottomrule
\end{tabularx}
\label{tab:PlatePositions}
\end{center}
\end{table}


Since the extension survey only covers a small azimuthal range, we will add in quadrature as uncertainty the average radial difference from the VCNL survey, which is listed in the \textit{Difference Scan-VCNL} column in Table~\ref{tab:AvgDeviations_surveys}. 
But we will not use that metric on the mylar plate Q1L outer, where we cannot fully trust the VCNL sensors. In fact the discrepancy between the laser scan and the VCNL sensors is localized on the larger "waves" of the mylar (see pg.5 of~\cite{Kargiantoulakis:doc16324}). Previous analysis~\cite{Nguyen:doc4586} has shown that plate waviness has small effect on beam dynamics, as would be expected for small-scale ("high-frequency") structure, and therefore is not correct to assign as uncertainty for the average radial location. 
For that plate we assign \SI{0.44}{mm} uncertainty term, the largest entry for a known quad in that column.


For Q1S and Q4S which were reinstalled after the vacuum incident, our best estimation for their location comes from the VCNL survey. On Q1S the difference is not significant anyway, but on Q4S we finally remove the \SI{2}{mm} discrepancy that probably comes from a calculation error.
The baseline uncertainty of the average radial location will be \SI{0.7}{mm}, the largest discrepancy between the laser scan and extension probes which is found on Q1S. Of course the comparison is not applicable on Q1S as the laser scan data is no longer valid, but that is also why a largest uncertainty is justified.



Since the best information on the Q1S and Q4S plates comes from the VCNL survey, we also need to add uncertainty for the discrepancy between that survey and the laser scan dataset.
We will use \SI{0.5}{mm}, the largest deviation on an individual plate (found on Q1S outer, excluding the miscalculated Q4S plates and the Q1L outer mylar plate) from the \textit{Difference Scan-VCNL} column in Table~\ref{tab:AvgDeviations_surveys}.
We also assign uncertainty based on the procedure to fix the geometric factors needed to express the VCNL survey into storage ring coordinates. We essentially calibrated the survey results on the known plates, in order to extend them to the unknown ones.
The small RMS from Table~\ref{tab:AvgDr} is a good measure of the stability of this process. We conservatively use twice that RMS, \SI{0.5}{mm}, as uncertainty for extending the survey to the unknown plates.


The final results for quad plate position and associated uncertainty are quoted in Table~\ref{tab:PlatePositions}.









