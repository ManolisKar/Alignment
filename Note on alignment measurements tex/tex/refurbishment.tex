\section{\label{sec:refurb} Refurbishment}
\textcolor{orange}{Hogan \& Jason}
\medskip

\textcolor{red}{\textit{Reused and polished plates and cages.}}
\medskip

The BNL top and bottom plate standoffs were refurbished, where new standoffs were produced and used as necessary, see Fig.~\ref{fig:tbstandoffs}. All of the BNL outer and inner plate standoffs were replaced with new standoffs, see Fig.~\ref{fig:oistandoffs}.
\begin{figure}[]
	\centering
	\begin{subfigure}{\columnwidth}
		\begin{tikzpicture}
			\draw (0, 0) node[inner sep=0]{\includegraphics[width=0.95\columnwidth]{fig-IMG_7631.jpeg}};
			\draw (0, 0) node[rotate=45,opacity=0.4]{{\fontsize{45}{54}\selectfont\textcolor{red}{\textbf{Preliminary}}}}; % rule of thumb: \fontsize{size}{1.2*size}
		\end{tikzpicture}
		%\includegraphics[width=0.95\columnwidth]{fig-IMG_7631.jpeg}
		\caption{Top and bottom plate standoffs; the metal end-caps are glued to the Macor pieces before being used with the plates.}\label{fig:tbstandoffs}
	\end{subfigure}
	%add desired spacing between images, e. g. ~, \quad, \qquad, \hfill etc (or a blank line to force the subfigure onto a new line).
	\begin{subfigure}{\columnwidth}
		\begin{tikzpicture}
			\draw (0, 0) node[inner sep=0]{\includegraphics[width=0.95\columnwidth]{fig-IMG_7658.jpeg}};
			\draw (0, 0) node[rotate=45,opacity=0.4]{{\fontsize{45}{54}\selectfont\textcolor{red}{\textbf{Preliminary}}}}; % rule of thumb: \fontsize{size}{1.2*size}
		\end{tikzpicture}
		%\includegraphics[width=0.95\columnwidth]{fig-IMG_7658.jpeg}
		\caption{Outside (left) and inside (right) plate standoffs.}\label{fig:oistandoffs}
	\end{subfigure}
	\caption{Macor standoffs give the quadrupole plates mechanical support and electrical isolation.}\label{fig:standoffs}
\end{figure}

The BNL quadrupole plate leads were made as two pieces from aluminum tubes that use pin joints, where the pins are crimped into place. These BNL leads were replaced with new leads that have the same shapes, but which are made from aluminum rods, see Fig.~\ref{fig:plate_leads}. The bottom, outside, and top plate leads are now made as single pieces, while the inside plate leads are made as two pieces for ease of installation. The pins are now machined from the aluminum rod material, where the outside and inside plate leads also use weld joints. 
\begin{figure}[]
	\centering
	\begin{tikzpicture}
		\draw (0, 0) node[inner sep=0]{\includegraphics[width=0.95\columnwidth]{fig-IMG_7745.jpeg}};
		\draw (0, 0) node[rotate=45,opacity=0.4]{{\fontsize{45}{54}\selectfont\textcolor{red}{\textbf{Preliminary}}}}; % rule of thumb: \fontsize{size}{1.2*size}
	\end{tikzpicture}
	%\includegraphics[width=0.95\columnwidth]{fig-IMG_7745.jpeg}
	\caption{Short quadrupole plate leads: bottom plate (top), outside plate (second from the top), top plate (second from the bottom), and inside plate (bottom). Long quadrupole plate leads have the same basic shapes as the short leads, but with slightly different dimensions.}\label{fig:plate_leads}
\end{figure}

The BNL outer/inner plate end-curls were reused, where new end-curls were produced and used as necessary, see Fig.~\ref{fig:plate_end_curl}. The new end-curls are made from aluminum rods, where the pins are machined from the aluminum rod material.
\begin{figure}[]
	\centering
	\begin{tikzpicture}
		\draw (0, 0) node[inner sep=0]{\includegraphics[width=0.95\columnwidth]{fig-IMG_7663.jpeg}};
		\draw (0, 0) node[rotate=45,opacity=0.4]{{\fontsize{45}{54}\selectfont\textcolor{red}{\textbf{Preliminary}}}}; % rule of thumb: \fontsize{size}{1.2*size}
	\end{tikzpicture}
	%\includegraphics[width=0.95\columnwidth]{fig-IMG_7663.jpeg}
	\caption{A quadrupole plate end-curl along with a small sample of outer/inner plate. A curl is placed around the outer and inner plates to reduce the electric field at the edges, so as to reduce the rate of quadrupole sparks.}\label{fig:plate_end_curl}
\end{figure}

Macor support plates provide mechanical support and electrical isolation to quadrupole leads, see Fig.~\ref{fig:batman}. The support plate tabs that surround the leads require protection from trapped electrons, as these electrons travel along the direction of the leads. The tabs are protected by arc suppression rings, see Fig.~\ref{fig:buttons}, which cause the trapped electrons to travel around the tabs. The BNL support plates and arc suppression rings were refurbished, and new support plates and buttons were required for the use of quadrupole extensions, see Section~\ref{sec:exten}. Half of the support plates used BNL style arc suppression rings, and the other half of the plates used a new test design that was larger so that the tabs were completely covered. All of the test design arc suppression rings were removed during the 2018 summer shutdown and were replaced with BNL style arc suppression rings due to concerns over sparking between the rings and plates. 
\begin{figure}[]
	\centering
	\begin{subfigure}{\columnwidth}
		\begin{tikzpicture}
			\draw (0, 0) node[inner sep=0]{\includegraphics[width=0.95\columnwidth]{fig-IMG_7916.jpeg}};
			\draw (0, 0) node[rotate=45,opacity=0.4]{{\fontsize{45}{54}\selectfont\textcolor{red}{\textbf{Preliminary}}}}; % rule of thumb: \fontsize{size}{1.2*size}
		\end{tikzpicture}
		%\includegraphics[width=0.95\columnwidth]{fig-IMG_7916.jpeg}
		\caption{A Macor support plate is colloquially know as a ``batman''.}\label{fig:batman}
	\end{subfigure}
	%add desired spacing between images, e. g. ~, \quad, \qquad, \hfill etc (or a blank line to force the subfigure onto a new line).
	\begin{subfigure}{\columnwidth}
		\begin{tikzpicture}
			\draw (0, 0) node[inner sep=0]{\includegraphics[width=0.95\columnwidth]{fig-IMG_7579.jpeg}};
			\draw (0, 0) node[rotate=45,opacity=0.4]{{\fontsize{45}{54}\selectfont\textcolor{red}{\textbf{Preliminary}}}}; % rule of thumb: \fontsize{size}{1.2*size}
		\end{tikzpicture}
		%\includegraphics[width=0.95\columnwidth]{fig-IMG_7579.jpeg}
		\caption{Two types of arc suppression rings were used during the Commissioning Run and \runone: test (left) and BNL (right) design. Aluminum arc suppression rings are colloquially know as ``buttons''.}\label{fig:buttons}
	\end{subfigure}
	\caption{Macor support plates and aluminum arc suppression rings.}\label{fig:batman_buttons}
\end{figure}

The BNL customized standoff-plate screws were refurbished, where new screws were produced and used as necessary, see Fig.~\ref{fig:plate_screws}.
\begin{figure}[]
	\centering
	\begin{tikzpicture}
		\draw (0, 0) node[inner sep=0]{\includegraphics[width=0.95\columnwidth]{fig-IMG_8266.jpeg}};
		\draw (0, 0) node[rotate=45,opacity=0.4]{{\fontsize{45}{54}\selectfont\textcolor{red}{\textbf{Preliminary}}}}; % rule of thumb: \fontsize{size}{1.2*size}
	\end{tikzpicture}
	%\includegraphics[width=0.95\columnwidth]{fig-IMG_8266.jpeg}
	\caption{Screws used to attach standoffs to outer and inner plates: horizontal standoffs shown in Fig.~\ref{fig:oistandoffs} (left) and vertical standoffs shown in Fig.~\ref{fig:vertical_standoffs} (right). The screws on the right are used for the Mylar Q1 outer plate and vertical standoff upgrades, see Section~\ref{sec:mylar}. The screws heads are ground down to be flat so as to reduce the profile and minimize electric field distortions.}\label{fig:plate_screws}
\end{figure}

\medskip
\textcolor{red}{\textit{Do we want to say anything else about refurbishment???}}


